\documentclass[letter]{article} 
\usepackage{amsmath, amsthm, amssymb, amsfonts, geometry, import, mathrsfs, multirow, algorithm2e, gauss, tikz}
\geometry{left=2.5cm,right=2.5cm,top=2.5cm,bottom=2.5cm}
\newenvironment{solution}
	{\renewcommand\qedsymbol{$\square$}\begin{proof}[Solution]}
	{\end{proof}}
\author{Heming Han a.k.a. RealMonia} 
\title{\bf Abstract Algebra Definitions Checklist} 
\date{\today}
\begin{document}
\maketitle
\section{Basic}
\subsection{Properties for operations}
\begin{enumerate}
    \item Closure.
    \item Associative.
    \item Commutative.
    \item Unital (existence of identity)
    \item Distributive.
    \item Invertible.
\end{enumerate}
\section{Group-related}
\subsection{Group}
\subsection{Semi-group}
\subsection{Monoid}
\section{Ring-related}
\subsection{Ring}
\begin{enumerate}
	\item Definition 1\\
	A nonempty set R with two operations *(usually written as addition and multiplication) that satisfy the
	following axioms. For $\forall a,b,c\in R$:
	\begin{enumerate}
		\item If $a\in R$ and $b\in R$, then $a+b\in R$ (closure for addition).
		\item $a+(b+c)=(a+b)+c$ (Associative addition).
		\item $a+b=b+a$ (Commutative addition).
		\item There is an element $0_R$ in $R$ such that $a+0_R=a=0_R+a$ for every $a\in R$. (Additive identity or zero element).
		\item For each $a\in R$, the equation $a+x=0_R$ has a solution in $R$.
		\item If $a\in R$ and $b\in R$, then $ab\in R$ (closure for multiplication).
		\item $a(bc)=(ab)c$ (associative multiplication).
		\item $a(b+c)=ab+ac$ and $(a+b)c=ac+bc$ (distributivity).
	\end{enumerate}
	\item Definition 2\\
	A nonempty set R with addition and multiplication such that:
	\begin{enumerate}
		\item $(R,+)$ is an abelian group.
		\item $(R,\cdot)$ is a semigroup.
		\item $(R,+,\cdot)$ is distributive for addition and multiplication.
	\end{enumerate}
	\item Relative Extension
	\begin{enumerate}
		\item Commutative Ring: ring $R$ satisfies $ab=ba$ for $\forall a,b \in R$ (Commutative multiplication).
		\item Ring with identity: ring $R$ that contains an element $1_R$ satisfying $a1_R=a=1_Ra$ for $\forall a\in R$ (multiplicative identity).
	\end{enumerate}
\end{enumerate}


\subsection{Subring}
\begin{enumerate}
	\item When a subset $S$ of a ring $R$ is itself a ring under the addition and multiplication in $R$, then we say that $S$ is a subring of $R$.
	\item $R$ is a ring and $S$ is a subset of $R$ such that
	\begin{enumerate}
		\item S is closure under addition and multiplication.
		\item $0_R\in S$.
		\item If $a\in S$ then the solution of the equation $a+x=0_R$ in $S$ (addition inverse exists).
	\end{enumerate}
	Then $S$ is a subring of $R$.
\end{enumerate}

\subsection{Integral Domain}
An integral domain is a commutative ring $R$ with identity $1_R\neq 0_R$ such that:

whenever $a,b\in R$ and $ab=0_R$ then $a=0_R$ or $b=0_R$.	
\section{Field-related}
\subsection{Field}
A field is a commutative ring $R$ with identity $1_R\neq 0_R$ such that:

For each $a\neq 0_R$ in $R$, the equation $ax=1_R$ has a solution in $R$. 
\section{Others}
\begin{enumerate}
    \item 
\end{enumerate}

\end{document}
